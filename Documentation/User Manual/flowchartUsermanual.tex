\documentclass[11pt,a4paper,titlepage]{article}
\usepackage{graphicx}
\DeclareGraphicsExtensions{.png}
\DeclareGraphicsExtensions{.jpg}

\begin{document}
\begin{titlepage}
	\begin{center}
		
		\begin{figure}[t]
			\centering
			\includegraphics[width=350px]{UP_Logo.png}
		\end{figure}	
	
	\begin{flushright} 
		
		\textbf{\LARGE COS301 Main Project}
		\newline \newline \newline
		\textbf{\LARGE Flowchart Simulation Tool User Manual}
		\newline \newline \newline
 		\textbf{\LARGE ThinkTech}
		\newline \newline \newline
	\end{flushright}
	
	\begin{flushright} \large
			
			\textbf{\LARGE Stakeholders}\newline 
			Lelethu Zazaza 13028023\newline
			Goodness Adegbenro 13046412\newline
			Hlavutelo Maluleke 12318109\newline
			Tshepiso Magagula 12274195\newline
			Xoliswa Ntshingila 13410378\newline
			
			
	\end{flushright}
	
	\begin{flushright} \large
	
			\textbf{\LARGE Client}\newline 
			Mr Willem S. van Heerden \newline
			University of Pretoria (Junior Lecturer)\newline
			
	\end{flushright}
		
		\vspace{1 cm}
		

		
		\vfill
		
		{\LARGE Version 0.2}
		\\
		{\large \today}		
		
		
	\end{center}
\end{titlepage}


\newpage
\tableofcontents
\pagenumbering{roman}
\newpage
\pagenumbering{arabic}
\section{General Information}
	\subsection{System Overview}
	
		Flowchart Simulation Tool is an application which allows for 				designing and executing flowchart diagrams. This application is 		intended for academic purposes for students with basic knowledge of 		programming design and implementation. This is a platform where the students can construct their own flowcharts, test and use them as they would like.
	
\section{System Summary}
	\subsection{System Configuration}
		
		The flowchart tool is intended to operate on any Linux 						distribution. However, it can also be used on Windows and 			Mac OS. It is not a very sophisticated software, not so many configurations required.
		
		Any version of Java should be installed to allow for successful execution of the program. 
		
		\subsection{System Installation}
		
		This system is a "plug-and-play" type of application, for which the executable file can be found from the University of Pretoria (Computer Science) website.
	
\section{Getting Started}
	
	For one to get to use the system, that particular user must be registered at the University of Pretoria, and have access to the Linux platform which is granted by the Department of Computer Science. From Linux, the user must be able to access the application, without any further installation or signing-up required.
	
	\subsection{Running Software}
		The application will not require any installation. To run the 			application the user will only require the executable file which will be provided via the on-line University of Pretoria (Computer Science) website. With the executable file the user will only need to double click on the file, then the system will star-up.
		
		\subsection{Software Layout}
		The layout is composed of the canvas, flowchart tools and menu options. \newline
		
		\begin{itemize}
			\item \textbf{Canvas:} This space is provided to design a 			well-formed flowchart. Components will be dragged from the flowchart 				tools menu consisting of available components and dropped onto the 				canvas. See figure below.
			
			\item \textbf{Flowchart Tools:} This contains all the tools 				required to construct the flowchart. All the components will be 			dragged and dropped onto the canvas. Flowchart Tools also includes 				buttons to run the simulation step-by-step and start-to-end. See figure below.
			
			\item \textbf{Menu Options -} The menu option provides options of 				creating, saving, deleting and loading projects onto the canvas. See figure below.
			
		\end{itemize}
	
	\includegraphics[width=\textwidth]{ManualDemo.jpg}
	\newpage
\section{Using the System}
	\subsection{Creating New Project}
	
		To create a new project select "File" in the menu bar and a drop down menu will appear. Select the "Create New Project" option, a pop-up window will appear requesting the desired file name for the project.
		
		
	\subsection{Adding Components to Canvas}
	
	Adding a component to the canvas is a simple as dragging and dropping a component from the Flowchart Tools. Select the component that you wish to add to the canvas and then drag it to the desired location on the canvas and then drop it.
	
	\subsection{Editing Component}
	
	Editing the component entails manipulation of the features and adding code inside the component. To edit the component double-click on the component and the a pop-up window will appear with the options of changing features or adding code.
		
	\subsection{Removing Components from Canvas}
	
	To remove any component from the canvas right-click on the component and select delete component. To remove multiple components hold the "SHIFT" key and select the components and then right-click and select delete component.
		
	\subsection{Saving Project}
	
	To save your current progress go to "File" in the menu bar, in the drop down menu select "Save". Enter the name of the file and then the project will be saved in a directory for flowchart projects.
	
	\subsection{Run Simulation}
	To run the flowchart select the "Run" icon on the flowchart tools window and the the whole flowchart will execute. To run the flowchart step-by-step select the "Step" icon.
	
		
	\subsection{Opening Existing Project}
	
	To open an existing project go to the menu bar and select the "File" option and then select the "Open Project" menu item. Search for an existing flowchart project with the valid extention. The flowchart project will now be loaded onto the canvas and should be able to be updated.
	
\section{Troubleshooting}

This system is not connected to any other external systems, this reduces the amount of errors that might usually breakdown the system.

\end{document}