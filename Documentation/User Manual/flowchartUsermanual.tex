\documentclass[11pt,a4paper,titlepage]{article}
\usepackage{graphicx}
\usepackage[a4paper, total={6in,8in}]{geometry}
\DeclareGraphicsExtensions{.png}
\DeclareGraphicsExtensions{.jpg}

\begin{document}
\begin{titlepage}
	\begin{center}
		
		\begin{figure}[t]
			\centering
			\includegraphics[width=350px]{UP_Logo.png}
		\end{figure}	
	
	\begin{flushright} 
		
		\textbf{\LARGE COS301 Main Project}
		\newline \newline \newline
		\textbf{\LARGE Flowchart Simulation Tool User Manual}
		\newline \newline \newline
 		\textbf{\LARGE ThinkTech}
		\newline \newline \newline
	\end{flushright}
	
	\begin{flushright} \large
			
			\textbf{\LARGE Stakeholders}\newline 
			Lelethu Zazaza 13028023\newline
			Goodness Adegbenro 13046412\newline
			Hlavutelo Maluleke 12318109\newline
			Tshepiso Magagula 12274195\newline
			Xoliswa Ntshingila 13410378\newline
			
			
	\end{flushright}
	
	\begin{flushright} \large
	
			\textbf{\LARGE Client}\newline 
			Mr Willem S. van Heerden \newline
			University of Pretoria (Junior Lecturer)\newline
			
	\end{flushright}
		
		\vspace{1 cm}
		

		
		\vfill
		
		{\LARGE Version 0.2}
		\\
		{\large \today}		
		
		
	\end{center}
\end{titlepage}


\newpage
\tableofcontents
\pagenumbering{roman}
\newpage
\pagenumbering{arabic}
\section{System Overview}
	
		Flowchart Simulation Tool is an application which allows for designing and executing flowchart diagrams. A flowchart is a type of diagram that represents an algorithm, workflow or process, showing the steps as boxes of various kinds, and their order by connecting them with arrows (https://en.wikipedia.org/wiki/Flowchart). This diagrammatic representation illustrates a solution model to a given problem. Flowcharts are used in analysing, designing, documenting or managing a process or program in various fields. \newline
		
		This application is intended for academic purposes for first students with basic knowledge of programming design and implementation. This is a platform where the students can construct their own flowcharts, test and use them as they would like. \newline
		
		Furthermore, this system is highly related to the already existing edition of JavaBlock. The idea behind this system and that of JavaBlock is the same, they both pursue the same purpose. However, they are different when it comes to the usability of the system. With Flow (this system) focusing more on the interaction the user has on the system. For example, with JavaBlock to place a block onto the canvas one requires a couple of operations but with Flow, drag-and-drop operation is used.\newline
		
		Most importantly, this system is solely designed for the benefits of first-year students. However, it can also be extended for other relevant uses. For all other users, the purpose is still the same as other flowchart interpreters.
		
\section{System Configuration}
		
		This is desktop application, it is intended to operate on any Linux distribution. However, it can also be used on Windows platform. It is not a very sophisticated software and since the system is a minimal stand-alone application, not so many configurations required. \newline
		
		It is so far recommended that the computer in which the system will be executed on have Java 7 and above installed to allow the successful execution of the program. The system was not tested on Windows XP and the versions released before that, it is then recommended that the desktops should be running Windows 7 and above.\newline		
		
\section{System Installation}

		The client is responsible for the distribution of the software, it will be available in the University of Pretoria - Computer Science website. The compressed folder will contain the installation file for Windows machines, and an executable .jar file for Linux machines. \newline
		
		This system requires a complete installation in order to fully operate it. For different platforms, various installation files are required. \newline \newline
		For machines running on Windows platform, an executable is provided in the link below. However, for machines running on Linux machines, and executable .jar file is provided.\newline
		
		

		\subsection{Windows installation}
		
		The image below shows the file that one requires for a complete successful installation of the program. Double-click the file to begin the installation process. \newline \newline \newline
		\includegraphics[width=13cm, height=2cm]{images/Install1.jpg}		
		\begin{center}
		Figure: The executable file to install Flow \newline
		\end{center}
			
		Next, follow the installation prompts and complete the installation. After a successful installation the application will be added to computer programs and it is accessible from the start menu. \\ \\
		
		\includegraphics[width=13cm]{images/Install2.jpg}		
		\begin{center}
		Figure: Follow Windows installation instructions
		\end{center}
		
		\subsection{Linux installation}
		
		All necessary files required to run this system on Linux platform are provided in the download link provided above. In the Linux folder, open "dist" folder and double-click the Flow.jar file. It requires no installation.\newline 
		
		This is different from Windows, where the user has to go through the installation process. All that is required on Linux is the files mentioned above.\newline \newline
		
		\includegraphics[width=14cm]{images/linuxInstall.jpg}		
		\begin{center}
			Figure: Opening Flow in Linux \newline
		\end{center}
		
		After clicking the executable .jar file, a splash will show before the actual application. \newline \newline
		\includegraphics[width=14cm]{images/splash.jpg}		
		\begin{center}
			Figure: Flow splash \newline
		\end{center}
		
		Then the actual Flow in Linux will start running, in the same way as on Windows but now on Linux.\newline
		
		\includegraphics[width=14cm]{images/linuxFlow.jpg}		
		\begin{center}
			Figure: Flow in Linux platform \newline
		\end{center}
		

\section{Getting Started}
	
	% For one to get to use the system, that particular user must be registered at the University of Pretoria, and have access to the Linux platform which is granted by the Department of Computer Science. From Linux, the user must be able to access the application, without any further installation or signing-up required.
	
	The sole purpose of this software was for first year students registered with the University of Pretoria to use as a flowchart simulator tool. However, it is also available for everyone else who wishes to use it for various purposes. Instruction on how to access the necessary files to install the software are provided above.\\
	\\For University of Pretoria students, the system will be installed in the Informatorium labs, they will be able to use it from the computers available in the Informatorium. Access to the computers in the Informatorium is restricted to specific students, these students are ones to have access to the system.\\

% This section forms the bulk of the user manual.


	
	\subsection{Running Software}
	
	% Screenshots of how the file should be exucuted	
	
		% The application will not require any installation. To run the application the user will only require the executable file which will be provided via the on-line University of Pretoria (Computer Science) website. With the executable file the user will only need to double click on the file, then the system will star-up.
		
	%	For the time being an executable .jar file has been created to allow instant execution of the system without any installation. Below its a simple outline which illustrate this, an executable is clicked on the top right of the screen and Flow appears instantly. \newline \newline
	
	
		After a successful installation of Flow, the potential user can then start the application normally on their machine. The image below depicts the desktop short-cut icon created after the installation is complete.\newline \newline
		
		\includegraphics[width=14cm]{images/DesktopIcon.jpg}
		\begin{center}
			Figure: Flow desktop icon.\\
		\end{center} 
		
		
		Double-clicking the icon on the desktop will consequently open the application. More about the application is discussed below.
		
		
		\subsection{Software Layout}
		The layout is composed of the canvas, flowchart tools and menu options. The figure below depicts the general layout of the entire system.
		
		\includegraphics[width=\textwidth]{images/SystemLayout.jpg}
		\begin{center}
		Figure: The General System Layout.\newline
		\end{center}
		
		
		\begin{itemize}
			\item \textbf{Canvas:} This space is provided to design a 			well-formed flowchart. Components will be dragged from the flowchart 				tools menu consisting of available components and dropped onto the 				canvas. See figure below.\newline
			There are components which are already implemented at start of the system, the 'start' and 'end' blocks are initialised by default on the canvas.\\ \\
			\includegraphics[width=14cm]{images/Canvas.jpg}
			\begin{center}
		Figure: The Canvas.\newline \newline 
		\end{center}
			
			\item \textbf{Flowchart Tools:} This contains all the tools required to construct the flowchart. All the components will be dragged and dropped onto the canvas (as illustrated by the the figure below), with the tools on the left and the canvas on the right. All the necessary components to build a flowchart are listed on the left toolbar.\newline
			
		The tools are draggable, i.e, from the toolbar hover over the block you want to use and drag it over to the canvas (the white plane on the left - as illustrated above). Under "using the system" is a detailed illustration on how to use these tools. \newline \newline			
			
			\includegraphics[width=14cm]{images/Tools.jpg}
			\begin{center}
		Figure: Flowchart Tool Components.\newline
		\end{center} 
			
						
			\item \textbf{Menu Options -} The menu option provides options of creating, saving, deleting and loading projects onto the canvas. Also, this provides the user with many more options which are accessible by hovering over the name of the option. The figure below shows the functionality of all this.\newline 
		
		\begin{itemize}
		\item New: The detailed description of this is given under "Creating New Flowchart"  below.
		\item Save:  The detailed description of this is given under "Saving The Flowchart" below.
		\item Delete: This will delete the entire project, from the main flowchart to the modules that the flowchart uses.
		\item Open: The detailed description of this is given under "Opening an existing flowchart" below.
		\item Show/Hide Grid: Shows or hides the grids on the canvas.
		\item Show/Hide Console: Shows or hides the execution console.
		\item Export as image: This captures the current screen of the flowchart and save it as image.
		\item Exit: Closes the application.\newline
		\end{itemize}				
			
		\includegraphics[width=14cm]{images/Menu.jpg}
		\begin{center}
			Figure: Menu Bar. \newline \newline \newline
		\end{center}
			
			
		\end{itemize}
	
	
%%%%%%%%%%%%%%%%%% Very important %%%%%%%%%%%%%%%%%%%%%%%%%%%%%%%%%%%%%%
	

\section{Using the System}
	\subsection{Creating New Flowchart}
	
	When the application is opened for the first time, a default project is created with a default flowchart, which compiles and execute without any output generated. The user then will start building the flowchart from already generated default flowchart. This is also demonstrated by the figure on general flowchart layout. \newline\newline
However, the user can create new flowchart from the menu bar. To create a new flowchart select "File" in the menu bar (as illustrated by the figure below) and a drop down menu will appear.\newline \newline

		\includegraphics[width=14cm]{images/Menu.jpg}
		\begin{center}
			Figure: Creating new flowchart \newline \newline
		\end{center}

 Select the "New" option, a pop-up window will appear requesting the user to save the current flowchart. After choosing any of the options a new flowchart will be created on the canvas with a default flowchart as shown by the figure below.\newline \newline
 
 		\includegraphics[width=14cm]{images/newFlowchart.jpg}
		\begin{center}
			Figure: The newly created flowchart\newline \newline
		\end{center}
		
		
	\subsection{Adding Components to Canvas}
	
	The component section of Flow is divided into a couple of sections, flowchart symbols, loop structures and input/output blocks.
	\subsubsection{Flowchart symbols}
	
	To add any of these components onto the canvas, click on the component of choice and move it onto the preferred destination on the canvas. To move the selected component, hover over the canvas and right-click the component to drop it. figure below shows the selected block that is to be dropped onto the canvas.\newline
	
	\includegraphics[width=12cm]{images/addBlock.jpg} \newline
	
	After clicking on the block, it is moved to the required position on the canvas. Click on the canvas to place the block.
	
	
	
	\subsubsection{Loop structures}
	
	Loop structures are pre-defined loop components that can be added to the flowchart without any major drawing of the actual components. To add these components, click on the desired block, a pop-up will appear.\newline \newline
	
	\includegraphics[width=12cm]{images/loops.jpg} \newline
	
	On the space provided, the user can initialise the variable that is to be used in the loop, and also provide a loop condition. When all these values have been entered, click Ok to create a loop structure.
	
	\subsubsection{Input/Output blocks}	
	
	To add the input/output block(s), click on the block highlighted with fading words "input"/"output" inside the block. Then drop it on the position of your own choice by clicking on it again.\newline \newline
	
	\includegraphics[width=12cm]{images/io.jpg} \newline
	
	%\newline \newline
	\subsection{Editing Component}
	
	Editing the component entails manipulation of the features and adding code inside the component. To edit the component click on the component that you want to edit and the a pop-up window will appear on the far right side of the canvas with the option of adding simple code into the component. 
	
	\subsubsection{Declaration block}
	
	To edit a declaration block, click on it, then a pop-up window will appear with an option to add all the declarations that are going to be used throughout the flowchart.\newline \newline
	
	\includegraphics[width=12cm]{images/editDeclaration.jpg} \newline \newline
	On the pop-up window, the user can select with a drop-down menu the type of a variable to be declared. See figure below.\newline \newline
	
	\includegraphics[width=12cm]{images/editDeclaration1.jpg} \newline \newline
	Then in the box next to the drop-down menu, key in any variable name of your choice and initialise it to the value of your choice, then click on 'Add' button.\newline \newline
	
	\includegraphics[width=12cm]{images/editDeclaration2.jpg} \newline \newline
	The declaration block will not be automatically updated, to update the block, click on any blank space on the canvas then click on the block. After that, the block will be updated. \newline \newline
	
	\includegraphics[width=12cm]{images/editDeclaration3.jpg}
	
	
	\subsubsection{Processing block}
	
	To edit the processing block, click on it then a pop-up window will appear with a plain environment to allow the user to enter the necessary code into the block.\\ \newline
	\includegraphics[width=12cm]{images/editProcessing.jpg} \newline
	
	\subsubsection{Decision block}
	
	The process of editing the decision block is similar to that of editing a processing block. 
	
	\subsubsection{Module block}
	
	A module is edited in a similar manner as the processing block, the only difference being that in the code section the user enters the name of the module/function to be called. For example, look at the figure below. \newline \newline
	
	\includegraphics[width=12cm]{images/editModule.jpg} \newline
	
	\subsubsection{Loop structures}
	
	After creating a loop structure, each block that is contained in that structure is editable. That is, by clicking on any of the blocks the user is able to modify them. \newline \newline
	
	\includegraphics[width=12cm]{images/editLoop.jpg} \newline
	
	Each of these components are editable. Instructions on how to edit them have already been provided above, check for reference.\newline
	
	\subsubsection{Input block}
	
	By clicking on the input block from the canvas, the pop-up window requires the user to enter a "prefix", the prefix is used to prompt the user with a descriptive message. And also, the pop-up window requires the user to enter a "variable" that the value read is to be assigned to. \newline \newline
	
	\includegraphics[width=12cm]{images/editInput.jpg} \newline
	
	\subsubsection{Output block}
	
	The output block has a couple of components, these components are lifted below:
	\begin{itemize}
	\item Prefix: \newline
	The message to be printed before the variable is being printed out to the console.
	\item Variable: \newline
	The variable that the user wants to print to the console.
	\item Massage suffix: \newline
	The message to be printed after the variable has been printed out to the console. \newline
	\end{itemize}
		
	\includegraphics[width=12cm]{images/editOutput.jpg} \newline
	
	\subsection{Connecting blocks}
	
	To connect one block to another, click on the first block and hold "ctrl" key on the keyboard, then click the second block. More than two blocks can be connected in a similar fashion, initially click on the first block and hold ""ctrl" button, then click on as many blocks as you wish to connect in the order of flow.
				
	\subsection{Removing Components from Canvas}
	
	To remove any component from the canvas right-click on the component and select "Delete Block" as illustrated by the figure below. To remove multiple components hold the "SHIFT" key and select the components and then right-click and select delete component.\newline
	
	Alternatively, the user can select the block that he/she wishes to delete and press the "delete" key on the keyboard. \newline
	
	\includegraphics[width=12cm]{images/deleteComponent.jpg} \newline
		
	\subsection{Saving The Flowchart}
	
	To save your current progress go to "File" in the menu bar, in the drop down menu select "Save". Then a pop-up window will appear, browse the location where you would like to save the file to, enter the name of the file and make sure the file extension is .flow. Click save. \newline \newline
	
	\includegraphics[width=14cm]{images/saveFile.jpg}
	
	\subsection{Run Simulation}
	To run the flowchart select the "Play" icon on the flowchart tools window, this will highlight the other three buttons, execute the whole flowchart, step-by-step execution and the stop icons.\newline \newline
	
	\includegraphics[width=14cm]{images/runSimulator.jpg}\newline\newline
	\includegraphics[width=14cm]{images/runSimulator2.jpg}
		
	\subsection{Opening an Existing Flowchart}
	
	To open an existing flowchart go to the menu bar and select the "File" option and then select the "Open" menu item. Browse the files system for the that is to be opened, recall that only files with extension .flow can be opened. \newline
	
	\includegraphics[width=14cm]{images/openFlowchart.jpg}
	
\section{Troubleshooting}

Recall from "System Configuration" the Java version compatible with this system. Older versions of Java may cause the system to break, and not execute as expected. Hence, violating the reliability of the entire system. \newline

To get the latest version of Java go to https://java.com/en/download/ \newline \newline

\includegraphics[width=14cm]{images/installJava.jpg} \newline

Click on the download button, save the file to your local directory. Follow the installation instructions.

This system is not connected to any other external systems, this reduces the amount of errors that might usually breakdown the system.

\end{document}